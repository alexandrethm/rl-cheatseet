\section{RL Framework}

\paragraph{Markov Decision Process (MDP)}
A Markov Decision Process is a tuple $(S, A, R, P, \rho_0)$ where 
\begin{itemize}
    \item $S$ is the set of all valid states
    \item $A$ is the set of all valid actions
    \item $R: S \times A \times S \rightarrow \mathbb{R}$ is the \emph{reward function}, such that $r_t = R(s_t, a_t, s_{t+1})$. In the case the reward is stochastic and $r_t$ is a random variable, we have $R(s, a, s') = \mathbb{E}[r_t \mid s_t=s, a_t=a, s_{t+1}=s']$
    \item $P: S \times A \times S \rightarrow [0, 1]$ the \emph{transition probability function}, such that $P(s' \mid s, a)$ is the probability of transitioning into state $s'$ if you are in state $s$ and take action $a$
    \item $\rho_0: S \rightarrow [0, 1]$ is the starting state distribution
\end{itemize}

\paragraph{Markov property} Transitions only depend on the most recent state and action, and no prior history : $P(s_{t+1}\mid s_t,a_t,\dots,s_1,a_0,s_0) = P(s_{t+1}\mid s_t,a_t)$. This assumption does not always hold, for instance when the observed state does not contain all necessary information (\emph{Partially Observable Markov Decision Process}), or when $P$ and $R$ actually depend on $t$ (\emph{Non-Stationary Markov Decision Process}).

When the MDP is known (e.g. small tabular environments), optimal policies can be found offline without interacting with the environment, using \nameref{section:dp} (DP) algorithms. But this is generally not the case and RL algorithms have to do trial-and-error search (like bandits problems), and have to deal with \emph{delayed rewards}. Actions may affect not only the immediate reward, but also the next state and therefore all subsequent rewards.

\paragraph{Definitions}
\begin{itemize}
    \item The \emph{policy} $\pi$ determines the behavior of our agent, who will take actions $a_t \sim \pi(\cdot \mid s_t)$. Policies can be derived from an action-value function or can be explicitly parameterized and denoted by $\pi_{\theta}$. They can also be deterministic, in which case they are sometimes denoted by $\mu_{\theta}$, with $a_t = \mu_{\theta}(s_t)$.
    \item A \emph{trajectory} $\tau = (s_0, a_0, s_1, \dots)$ is a sequence of states and actions in the world, with $s_0 \sim \rho_0$ and $s_{t+1} \sim P(\cdot \mid s_t, a_t)$. It is sampled from $\pi$ if $a_t \sim \pi(\cdot \mid s_t)$ for each $t$. Trajectories are also called \emph{episodes}.
    \item The return $R(\tau)$ is the cumulative reward over a trajectory and is the quantity to be maximized by our agent. It can refer to the \emph{finite-horizon undiscounted return} $R(\tau) = \sum_{t=0}^T r_t$ or the \emph{infinite-horizon discounted return} $R(\tau) = \sum_{t=0}^{\infty} \gamma^t r_t$ for instance. Parameter $\gamma$ is called the \emph{discount factor}.
    \item The \emph{on-policy value function}: $V^{\pi}(s) = \E_{\tau \sim \pi}[R(\tau) \mid s_0=s]$
    \item The \emph{on-policy action-value function}: $Q^{\pi}(s,a) = \E_{\tau \sim \pi}[R(\tau) \mid s_0=s, a_0=a]$. We have $V^{\pi}(s) = \E_{a \sim \pi(\cdot \mid s)}[Q^\pi(s,a)]$.
    \item The \emph{advantage function}: $A^\pi(s,a) = Q^\pi(s,a) - V^\pi(s)$
    \item The optimal value and action-value functions are obtained by acting according to an optimal policy $\pi^*$: $V^*(s) = \max_{\pi} V^{\pi}(s)$, $Q^*(s,a) = \max_{\pi} Q^{\pi}(s,a)$. We have $V^*(s) = \max_a Q^*(s,a)$, and a deterministic optimal policy can be obtained with $\pi^*(s) = \argmax_a Q^*(s,a)$.
\end{itemize}

\paragraph{Bellman Equations}
\begin{equation}
    % V^{\pi}(s) = \E_{a \sim \pi(\cdot \mid s), s' \sim P}[R(s, a, s') + \gamma V^{\pi}(s')]
    V^{\pi}(s) = \E_{\substack{a \sim \pi(\cdot \mid s) \\ s' \sim P}}[R(s, a, s') + \gamma V^{\pi}(s')]
\label{eq:bellman-v}
\end{equation}

\begin{equation}
    Q^{\pi}(s, a) = \E_{s' \sim P}\left[R(s, a, s') + \gamma \E_{a'\sim \pi(\cdot \mid s')}[Q^{\pi}(s', a')] \right]
\label{eq:bellman-q}
\end{equation}

\begin{equation}
    V^*(s) = \max_a \E_{s' \sim P}\left[R(s, a, s') + \gamma V^*(s')\right]
\label{eq:bellman-v*}
\end{equation}

\begin{equation}
    Q^*(s, a) = \E_{s' \sim P}\left[R(s, a, s') + \gamma \max_{a'} Q^*(s', a')\right]
\label{eq:bellman-q*}
\end{equation}

Where $s' \sim P$ is a shorthand for $s' \sim P(\cdot\mid s,a)$